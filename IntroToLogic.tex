\begin{section}{Introduction to Logic}

After diving in head first in the last section, we'll take a step back and do a more careful examination of what it is we are actually doing.

\begin{definition}
A \textbf{proposition} (or \textbf{statement}) is a sentence that is either true or false.
\end{definition}

For example, the sentence ``All liberals are hippies'' is a false proposition.  However, the perfectly good sentence ``$x=1$'' is \emph{not} a proposition all by itself since we don't actually know what $x$ is.

\begin{exercise} Determine whether the following are propositions or not. Explain.
\begin{enumerate}
\item All cars are red.
\item Van Gogh was the best artist ever. 
\item If my name is Joe, then my name starts with the letter J.
\item If my name starts with the letter J, then my name is Joe.
\item $f$ is continuous.
\item All functions are continuous.
\item If $f$ is a differentiable function, then $f$ is continuous function.
\item The president had eggs for breakfast the morning of his tenth birthday.
\item What time is it? 
\item There exists an $x$ such that $x^2=4$.
\item $x^2=4.$
\item $\sqrt{2}$ is an irrational number.
\item For all real numbers $x$, $x^3=x$.
\item There exists a real number $x$ such that $x^3=x$.
\item $p$ is prime.
\end{enumerate}
\end{exercise}

Given two propositions, we can form more complicated propositions using the logical connectives ``and", ``or", and ``If\ldots, then\ldots".

\begin{definition}
Let $A$ and $B$ be propositions.  The proposition ``\textbf{$A$ and $B$}'' is true if and only if both $A$ and $B$ are true.  The statement ``$A$ and $B$'' is expressed symbolically as 
\[
A \wedge B
\]
and is known as the \textbf{conjunction} of $A$ and $B$.
\end{definition}

\begin{definition}
Let $A$ and $B$ be propositions.  The proposition ``\textbf{$A$ or $B$}'' is true if and only if at least one of $A$ or $B$ is true.  The statement ``$A$ or $B$'' is symbolically represented as
\[
A \vee B
\]
and is known as the \textbf{disjunction} of $A$ and $B$.
\end{definition}

\begin{definition}
Let $A$ be a proposition.  The \textbf{negation} of $A$, denoted $\neg A$, is true if and only if $A$ is false.
\end{definition}

\begin{exercise}Describe the meaning of $\neg (A \wedge B)$ and $\neg (A \vee B)$.
\end{exercise}

\begin{definition}
A \textbf{truth table} is a table that illustrates all possible truth values for a proposition.  
\end{definition}

\begin{example}
Let $A$ and $B$ be propositions.  Then the truth table for the conjunction $A\wedge B$ is given by the following.
\[
\begin{tabular}{| c | c  | c |} \hline
$A$  &  $B$ & $A \wedge B$  \\ \hline\hline
T & T & T  \\ 
T & F & F  \\ 
F & T & F  \\ 
F & F & F  \\ \hline
\end{tabular}
\]
Notice that we have columns for each of $A$ and $B$.  The rows for these two columns correspond to all possible combinations for $A$ and $B$.  The third column gives us the truth value of $A\wedge B$ given the possible truth values for $A$ and $B$.
\end{example}

\begin{exercise}
Create a truth table for each of $A \vee B$, $\neg A$, $\neg (A \wedge B)$, and $\neg A \wedge \neg B$.  Feel free to add additional columns to your tables to assist you with intermediate steps.
\end{exercise}

\begin{exercise}
Suppose $P$ is a complex proposition built out of the propositions $A$, $B$, and $C$.  How many rows would the truth table for $P$ require?
\end{exercise}

\begin{definition}\label{def:conditional}
Let $A$ and $B$ represent propositions.  The conditional proposition ``\textbf{If $A$, then $B$}'' is expressed symbolically as 
\[
A \implies B
\]
and has the following truth table.
\[
\begin{tabular}{| c | c  | c |} \hline
$A$  &  $B$ & $A \implies B$  \\ \hline\hline
T & T & T  \\ 
T & F & F  \\ 
F & T & T  \\ 
F & F & T  \\ \hline
\end{tabular}
\]
\end{definition}

\begin{exercise}\label{exer:translations}
Let $A$ represent ``6 is an even number'' and $B$ represent ``6 is a multiple of 4.''  Express each of the following in ordinary English sentences and state whether the statement is true or false.
\begin{enumerate}
  \item $A \wedge B$
  \item $A \vee B$
  \item $\neg A$
  \item $\neg B$
  \item $\neg (A \wedge B)$
  \item $\neg (A \vee B)$
  \item $A \implies B$
\end{enumerate}
\end{exercise}

\begin{problem}
Suppose I am the coach of our co-ed dodgeball team and you all are the players.  I tell you ``If we win tonight, then I will buy you pizza tomorrow.''  After reviewing the definition of conditional proposition, determine the case(s) in which you can rightly claim to have been lied to.
\end{problem}

\begin{definition}
Two statements are \textbf{logically equivalent} (or \textbf{equivalent} if the context is clear) if and only if they have the same truth table.  That is, proposition $P$ is true exactly when proposition $Q$ is true, and $P$ is false exactly when $Q$ is false.  When $P$ and $Q$ are logically equivalent we denote this symbolically as 
\[
P \iff Q,
\]
which we read ``$P$ if and only if $Q$".  It is common to abbreviate ``if and only if'' as ``iff''.
\end{definition}

Each of the next three facts can be justified using truth tables.

\begin{theorem}
If $A$ is a proposition, then $\neg(\neg A)$ is equivalent to $A$.
\end{theorem}

\begin{theorem}\label{thm:demorgan}
If $A$ and $B$ are propositions, then $\neg(A \wedge B) \iff \neg A \vee \neg B$.  (\emph{Note}: This theorem is referred to as DeMorgan's Law.)
\end{theorem}

\begin{problem}
Let $A$ and $B$ be propositions.  Conjecture a statement similar to Theorem~\ref{thm:demorgan} for the proposition $\neg(A\vee B)$ and then prove it.
\end{problem}

\begin{definition}\label{def:converse}
The \textbf{converse} of $A \implies B$ is $B \implies A$.
\end{definition}

\begin{definition}
The \textbf{contrapositive} of $A \implies B$ is $\neg B \implies \neg A$.
\end{definition}

\begin{exercise}
Let $A$ and $B$ represent the statements from Exercise~\ref{exer:translations}.  Express the following in ordinary English sentences.
\begin{enumerate}
\item The converse of $A \implies B$
\item The contrapositive of $A \implies B$
\end{enumerate}
\end{exercise}

\begin{exercise} Find the contrapositive of the following statements: 
\begin{enumerate}
\item If $n$ is an even natural number, then $n+1$ is an odd natural number.\footnote{Despite the fact that each of ``$n$ is an even natural number" and ``$n+1$ is an odd natural number" are not propositions (since we cannot determine their truth values without knowing what $n$ is), the implication is a proposition.  We discuss this further when we introduce predicates.}
\item If it rains today, then I will bring my umbrella.
\item If it does not rain today, then I will not bring my umbrella.
\end{enumerate} \end{exercise}

\begin{exercise}
Provide an example of a true conditional proposition whose converse is false.
\end{exercise}

\begin{theorem}\label{thm:contrapos}
Assume $A$ and B are statements.  Then ${A\implies B}$ is equivalent to its contrapositive.
\end{theorem}

The upshot of Theorem~\ref{thm:contrapos} is that if you want to prove a conditional proposition, you can prove its contrapositive instead.  Prove each of the next two propositions using the contrapositive of the given statement.

\begin{theorem}
Assume $x$ and $y$ are integers.  If $xy$ is odd, then both $x$ and $y$ are odd. (Prove using contrapositive.)
\end{theorem}

\begin{theorem}
Assume $x$ and $y$ are integers.  If $xy$ is even, then $x$ or $y$ is even.  (Prove using contrapositive.)
\end{theorem}

\end{section}